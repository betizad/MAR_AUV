% !TeX spellcheck = en_US
\section{Introduction}

In the recent decades, the marine robotic community has  had an increasing interest to build vehicles for ocean exploration and exploitation. Remotely operated vehicles (ROV), autonomous surface/underwater vehicles (ASV/AUV), and gliders are fruitful products of these efforts. The vast majority of these
underwater robots are  propeller-driven, however, the efficiency of the currently adopted rotary propellers can
hardly reach even half of the propulsion efficiency of fish, that is about $90\%$ \cite{robotfish2015}.

The idea of building a robotic fish that is as efficient, agile, and quiet
as a natural fish is still far from becoming a reality. As stated  by 
the authors in \cite{robotfish2015}, in the next 10 years the possibility 
of a robot fish which will be able to swim with real fish in 
open water will get much more likely. Nevertheless, many research 
groups have studied aquatic locomotion and its implementation on a fish 
like robot. 
Including, the first robot fish from MIT \cite{barrett1988propulsive},  
the eel REEL robots  mimicking  anguilliform Locomotion \cite{McIsaac}, 
the lamprey like robot which is controlled through a neuromuscular interface \cite{Westphal2011}, 
a biorobotic platform
inspired by the lamprey \cite{Stefanini2012}, 
 the underwater snake robot Mamba developed at NTNU which propels using both aquatic locomotion and conventional propellers \cite{Liljeback2014}, and many more reported in \cite{robotfish2015}.


Taking inspiration from snakes and elongate fishes such as lampreys, 
Envirobot is an amphibious robot for outdoor robotic tasks. Its design 
is based on existing segmented anguilliform swimming robots, AmphiBot I 
\cite{crespi2005amphibot},  AmphiBot II \cite{crespi2006amphibot}, and 
AmphiBot III \cite{porez2014}. One of the main goals of the Envirobot 
project \cite{envirobot} is to design and construct an aquatic water 
sampling and water analysis robot, which can either work in a surveying 
mode according to a predefined path, or in autonomous-navigation mode, 
according to chemosensory and input from biological sensors; and that can 
store and/or communicate data analysis to an external observer. In 
autonomous surveying mode  the robot will sample and analyze waterbodies 
according to a predefined path and number of waypoints. During 
autonomous-navigation mode, the robot must guide its movements and sampling on the 
basis of the sensory input.  Autonomous-navigation is challenging but extremely 
useful, since Envirobot would be able to track and follow gradients 
of sparsely measured chemical pollution in waterbodies to find the 
source of pollution \cite{vergassola07,bayat2016}. 

The organization of the paper is as follows ...

